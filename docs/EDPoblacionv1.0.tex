\documentclass{article}
\usepackage{amsmath}
\usepackage[utf8]{inputenc}
\usepackage[spanish]{babel}
\begin{document}

\section*{Ejemplo}
Se sabe que la población de una comunidad crece con una razón proporcional al número de personas presentes en el tiempo $t$. Si la población inicial $P_0 = 1000$ se duplica en $3$ años, ¿en cuánto tiempo se triplicará, cuadruplicará y quintuplicará?

Definimos:
\begin{itemize}
  \item $P(t)$: población en el tiempo $t$,
  \item $P_0$: población inicial ($1000$),
  \item $k$: constante de proporcionalidad.
\end{itemize}

La ley de crecimiento proporcional se modela por la ecuación diferencial
\[
\frac{dP}{dt}=kP
\]
Separando variables e integrando:
\[
\frac{dP}{P}=k\,dt \quad\Rightarrow\quad \int\frac{dP}{P}=\int k\,dt
\]
\[
\ln|P|=kt+C \quad\Rightarrow\quad P(t)=Ce^{kt}
\]
Usando la condición inicial $P(0)=P_0$ obtenemos $C=P_0$, por lo tanto:
\[
P(t)=P_0 e^{kt}
\]

Como la población se duplica en $3$ años:
\[
2P_0 = P_0 e^{3k} \quad\Rightarrow\quad e^{3k}=2 \quad\Rightarrow\quad k=\frac{\ln 2}{3}
\]

#### Tiempo para triplicar ($P(t_3)=3P_0$):
\[
3P_0 = P_0 e^{k t_3} \quad\Rightarrow\quad e^{k t_3}=3
\]
\[
t_3=\frac{\ln 3}{k}=\frac{3\ln 3}{\ln 2}\approx 4.755\ \text{años}
\]

#### Tiempo para cuadruplicar ($P(t_4)=4P_0$):
\[
4P_0 = P_0 e^{k t_4} \quad\Rightarrow\quad e^{k t_4}=4
\]
\[
t_4=\frac{\ln 4}{k}=\frac{3\ln 4}{\ln 2}=\frac{3\cdot 2\ln 2}{\ln 2}=6\ \text{años}
\]

#### Tiempo para quintuplicar ($P(t_5)=5P_0$):
\[
5P_0 = P_0 e^{k t_5} \quad\Rightarrow\quad e^{k t_5}=5
\]
\[
t_5=\frac{\ln 5}{k}=\frac{3\ln 5}{\ln 2}\approx 6.97\ \text{años}
\]

### Código LaTeX completo
\begin{verbatim}
\documentclass{article}
\usepackage{amsmath}
\usepackage[utf8]{inputenc}
\usepackage[spanish]{babel}
\begin{document}

\section*{Ejemplo}
Se sabe que la población de una comunidad crece con una razón proporcional al número de personas presentes en el tiempo $t$. Si la población inicial $P_0 = 1000$ se duplica en $3$ años, ¿en cuánto tiempo se triplicará, cuadruplicará y quintuplicará?

Definimos:
\begin{itemize}
  \item $P(t)$: población en el tiempo $t$,
  \item $P_0$: población inicial ($1000$),
  \item $k$: constante de proporcionalidad.
\end{itemize}

La ley de crecimiento proporcional se modela por la ecuación diferencial
\[
\frac{dP}{dt}=kP.
\]
Separando variables e integrando:
\[
\frac{dP}{P}=k\,dt \quad\Rightarrow\quad \int\frac{dP}{P}=\int k\,dt
\]
\[
\ln|P|=kt+C \quad\Rightarrow\quad P(t)=Ce^{kt}.
\]
Usando la condición inicial $P(0)=P_0$ obtenemos $C=P_0$, por lo tanto
\[
P(t)=P_0 e^{kt}.
\]

Como la población se duplica en $3$ años:
\[
2P_0 = P_0 e^{3k} \quad\Rightarrow\quad e^{3k}=2 \quad\Rightarrow\quad k=\frac{\ln 2}{3}.
\]
\end{verbatim}